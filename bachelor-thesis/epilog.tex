\chapter*{Conclusion}

\noindent
To sum up, the main purpose of this thesis was to make a brief introduction to the field of planning and to explore some aspects that can be later employed in other types of related work. Starting from classical planning which can be expressed in different ways and lasting with hierarchical planning, HTN to be precise. The theory of HTN is not yet unified, therefore we can find different definitions and understandings about this topic in various publications. 

\medskip\noindent
One of the goals was to set proper boundaries via combinations of definitions from varying sources. By doing so, we could describe, compare, and analyze HTN semantics with the subtle goal of handling empty methods that are not handled accurately in plenty of similar papers. Difficulties start to appear if we want to use constraints that are bound to states. Also, new types of HTN semantics were introduced.

\medskip\noindent
In the last chapter of the thesis, we tried to find transformations of HTN models that might be suitable within some context. Most transformations were inspired by the theory of Automata and Grammars, yet they need a significant amount of extension as HTN models allow partially-ordered domains and state-constraints. 

\medskip\noindent
\xxx{TODO k cemu to je a jakym smerem pokracovat!!!!!!!!!!!!!!!!!!!!!}

\addcontentsline{toc}{chapter}{Conclusion}
