%%% Please fill in basic information on your thesis, which will be automatically
%%% inserted at the right places.

% Type of your thesis:
%	"bc" for Bachelor's
%	"mgr" for Master's
%	"phd" for PhD
%	"rig" for rigorosum
\def\ThesisType{bc}

% Language of your study programme:
%	"cs" for Czech
%	"en" for English
\def\StudyLanguage{cs}

% Thesis title in English (exactly as in the official assignment)
% (Note: \xxx is a "ToDo label" which makes the unfilled visible. Remove it.)
\def\ThesisTitle{Semantics and transformations of HTN models}

% Author of the thesis (you)
\def\ThesisAuthor{David Kroupa}

% Year when the thesis is submitted
\def\YearSubmitted{2024}

% Name of the department or institute, where the work was officially assigned
% (according to the Organizational Structure of MFF UK in English,
% see https://www.mff.cuni.cz/en/faculty/organizational-structure,
% or a full name of a department outside MFF)
\def\Department{Department of Theoretical Computer Science and Mathematical Logic}

% Is it a department (katedra), or an institute (ústav)?
\def\DeptType{Department}

% Thesis supervisor: name, surname and titles
\def\Supervisor{Prof. RNDr. Roman Barták, Ph.D.}

% Supervisor's department (again according to Organizational structure of MFF)
\def\SupervisorsDepartment{Department of Theoretical Computer Science and Mathematical Logic}

% Study programme (does not apply to rigorosum theses)
\def\StudyProgramme{Computer Science}

% An optional dedication: you can thank whomever you wish (your supervisor,
% consultant, who provided you with tea and pizza, etc.)
\def\Dedication{%
I am deeply grateful to my supervisor, Prof. RNDr. Roman Barták, Ph.D., for proposing such an interesting thesis topic that perfectly suited my interests, and for his invaluable guidance and support. \\

\noindent
I would also like to thank all the teachers who have taught me throughout my life, especially those at MFF UK. \\

\noindent
Last but not least, I want to express my sincere gratitude to my parents and close friends for their unconditional support.
}

% Abstract (recommended length around 80-200 words; this is not a copy of your thesis assignment!)
\def\Abstract{%
Hierarchical task network (HTN) is an approach to planning where compound tasks are decomposed into subtasks that can be either decomposed or executed right away. HTN is an extension of classical planning analogously as context-free languages extend regular languages. The goal of the thesis is split into two parts: describe, compare, and analyze various semantics and propose transformations of HTN models that do not lose any characteristics about the model. A significant piece of the thesis aims at handling empty methods which are usually not defined properly. The biggest challenges of HTN transformations lie in the proper management of constraints.
}

% 3 to 5 keywords (recommended) separated by \sep
% Keywords are useful for indexing and searching for the theses by topic.
\def\ThesisKeywords{%
hierarchical planning\sep hierarchical task networks\sep semantics\sep transformations
}

% If any of your metadata strings contains TeX macros, you need to provide
% a plain-text version for use in XMP metadata embedded in the output PDF file.
% If you are not sure, check the generated thesis.xmpdata file.
\def\ThesisAuthorXMP{\ThesisAuthor}
\def\ThesisTitleXMP{\ThesisTitle}
\def\ThesisKeywordsXMP{\ThesisKeywords}
\def\AbstractXMP{\Abstract}

% If your abstracts are long and do not fit in the infopage, you can make the
% fonts a bit smaller by this setting. (Also, you should try to compress your abstract more.)
\def\InfoPageFont{}
%\def\InfoPageFont{\small}  % uncomment to decrease font size

% If you are studing in a Czech programme, you also need to provide metadata in Czech:
% (in English programmes, this is not used anywhere)

\def\ThesisTitleCS{Sémantiky a transformace HTN modelů}
\def\DepartmentCS{Katedra teoretické informatiky a matematické logiky}
\def\DeptTypeCS{Katedra}
\def\SupervisorsDepartmentCS{Katedra teoretické informatiky a matematické logiky}
\def\StudyProgrammeCS{Informatika}

\def\ThesisKeywordsCS{%
hierarchické plánování\sep hierarchické sítě úloh\sep sémantika\sep transformace
}

\def\AbstractCS{%
Hierarchická síť úloh (HTN) je způsob plánování obsahující složené úlohy, které lze dekomponovat na podúlohy, které mohou být opět dekomponovány nebo přímo vykonány. HTN je rozšíření klasického plánování podobně jako bezkontextové jazyky rozšiřují regulární jazyky. Tato práce má dva cíle. Prvním je popsat, porovnat a analyzovat různé sémantiky. Druhým je navrhnout transformace HTN modelů, které by neztrácely žádné informace a vlastnosti modelů. Značná část této práce se zabývá vyřešením problémů prázdných metod, které většinou nejsou moc dobře zpracovány. Největší výzva HTN transformací je správná správa plánovacích omezujících podmínek.
}
